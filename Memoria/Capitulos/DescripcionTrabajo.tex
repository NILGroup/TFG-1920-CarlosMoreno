\chapter{Work Description}
\label{cap:descripcionTrabajo}

\section{Style Analyser}
In order to generate messages with the user's writing style, it is necessary to define parameters which will determine and describe it. For this purpose, we have developed a style analyser that extracts the messages written by the user and obtains the value of various metrics from them. Then it will be useful for analysing different user's emails and drawing conclusions about what parameters describe the writing style of each person more accurately. Besides most of the developed code will be reusable in the final application for analysing the user's messages.

In this section we are going to explain the architecture of this analyser (see \ref{ssection:stylearch}) and each of the modules that compose it (they are explained in sections \ref{ssection:extmod}, \ref{ssection:prepmod}, \ref{ssection:typomod} and \ref{ssection:measmod}). Finally, we are going to discuss the obtained results and analyse them for drawing a conclusion (this discussion can be looked up in \ref{ssection:resconc}).

\subsection{Architecture} \label{ssection:stylearch}
The first step when we are designing a system's architecture is to know its input and output. In this case, we want to implement a simple natural language processing system that analyses the writing style of emails. As we have previously mentioned, the writing style analysis will be represented through chosen metrics. Therefore, our system's output is going to be that chosen metrics (they are explained in section \ref{ssection:measmod}).

En respect of the system's input, because of the nature of the problem we face, it is reasonable to think that it must be a single email. However, we do not have the corpus of emails to analyse. For this reason, our first step will be to extract the emails that will be analysed. Hence, our system's input is going to be the Gmail user for accesing to the information that we are interested in.

\subsection{Extracting Module} \label{ssection:extmod}

\subsection{Preprocessing Module} \label{ssection:prepmod}

\subsection{Typographic Correction Module} \label{ssection:typomod}

\subsection{Measuring module} \label{ssection:measmod}

\subsection{Results and conclusions} \label{ssection:resconc}
