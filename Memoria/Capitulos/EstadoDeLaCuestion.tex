\chapter{State of the Art}
\label{cap:estadoDeLaCuestion}

\section{Gmail API}\label{sect:gmailapi}

\subsection{Messages resource}\label{ssect:msgres}
In most of the operations we are going to execute the correct management of messages will be essential. Therefore, knowing how the emails are represented in Gmail API and how to use them is imperative to understand how to work with this API. For this reason, in this section we are going to delve into the messages resource of the Gmail API, its structure and its methods. As we saw in Section \ref{ssect:userres}, we can access to this resource by invoking the \textit{messages()} method when we have a users resource.

\subsubsection{Resource representation}\label{sssect:msgresrep}

Regardless of which programming language is used, a messages resource \citep[/v1/reference/users/messages]{gmailAPI} internally has a dictionary structure as we can see down below:

\begin{python}
{
'id' : string,
'threadId' : string,
'labelIds' : [ string ],
'snippet' : string,
'historyId' : unsigned long,
'internalDate' : long,
'payload' : {
	'partId' : string,
	'mimeType' : string,
	'filename' : string,
	'headers' : [
		{
		'name' : string,
		'value' : string
		}
	],
	'body' : {
		'attachmentId' : string,
		'size' : integer,
		'data' : bytes
	},
	'parts' : [ (MessagePart) ]
	},
'sizeEstimate' : integer,
'raw' : bytes
}
\end{python}

The more important keys of this data structure for this work are:
\begin{itemize}
	\item\textit{id}: an immutable string which identifies the message.
	\item\textit{threadId}: we will explain the thread resource in Section \ref{ssect:threads} and we will see that a thread is composed of different messages that share common characteristics. The value of this field is a string which represent the identifier of the thread the message belongs to.
	\item\textit{labelIds}: a list of the identifiers of labels (see Section \ref{ssect:labres}) applied to the message.
	\item\textit{payload}: as we can see in the resource representation above, it has a dictionary data structure. The \textit{payload} field is the parsed email structure in the message parts. The more important keys of the \textit{payload} field are:
	\begin{itemize}
		\item\textit{mimeType}: the MIME type (see the explanation of \textit{Content-Type} header in Section \ref{sssect:mime}) of the message part.
		\item\textit{headers}: a list of headers. It contains the standard RFC 2822 \citep{rfc2822} email headers such as \textit{To}, \textit{From}, \textit{Subject} and \textit{Date}. Each header has a \textit{name} field, which is the name of the header (for example \textit{From}), and a \textit{value} field, which is the value of the header (following the same example as with the \textit{name} field: \textit{example@gmail.com} could be the value).
		\item\textit{parts}: a list which contains the different MIME message child parts (we will delve into this field in Section \ref{sssect:mime}).
		\item\textit{body}: a dictionary structure which contains the body data of this part (see Section \ref{sssect:mime}) in case it does not contain MIME message parts (otherwise it will be empty). This structure should not be confused with an attached file. Each MIME part contains a body property regardless of MIME type of the part.
	\end{itemize}
	\item\textit{raw}: the entire email message in an RFC 2822 \citep{rfc2822} formatted and base64url (see Section \ref{sssect:base64}) encoded string.
\end{itemize}

\subsubsection{Methods}\label{sssect:msgmethods}
As any other resource, the messages resource has different methods, many of whom we are going to need in the work. We will limit ourselves to describing the methods we may need to use:

\begin{itemize}
	\item\textit{attachments()}: returns the attachments resource (for more information about this resource, that we will not explain in detail, refer to \cite[/v1/reference/users/messages/attachments]{gmailAPI}).
	\item\textit{get(userId, id, format = 'full', metadataHeaders = None)}: if successful, this method returns the requested messages resource. Its parameters are:
	\begin{itemize}
		\item\textit{id}: the identifier string of the message we are looking for.
		\item\textit{userId}: the user's email address. As it happens with the \textit{getProfile} method of the users resource (see Section \ref{ssect:userres}), the special string value \textit{'me'} can be used to indicate the authenticated user.
		\item\textit{format} (optional parameter): the format in which we want the message returned. This field can take the following punctual values: \textit{'full'} (returns the entirely email data with body content parsed in the \textit{payload} messages resource field and the \textit{raw} field is empty), \textit{'metadata'} (returns only an email message with its identifier, email headers and labels), \textit{'minimal'} (returns only an email message with its identifier and labels) and \textit{'raw'} (returns the entirely email message data with the body content in the \textit{raw} messages resource field as a base64url (see Section \ref{sssect:base64}) encoded string and the \textit{payload} field is empty).
		\item\textit{metadataHeaders} (optional parameter): it is only used when the format parameter takes the punctual value of \textit{'metadata'}. It is a string list where we have to insert the headers we want to be included.
	\end{itemize}
	For knowing the required scopes for invoking this function refer to \cite[/v1/reference/users/messages/get]{gmailAPI}.
	\item\textit{list(userId, includeSpamTrash = false, labelIds = None, maxResults = None, pageToken = None, q = None)}: returns a resource with the following structure:
	\begin{python}
		{
		'messages' : [ users.messages resource ],
		'nextPageToken' : string,
		'resultSizeEstimate' : unsigned integer
		}
	\end{python}
	As it happens with the \textit{list} method of the labels resource (see Section \ref{ssect:labres}), \textit{'messages'} list does not contain all of a message information (for obtaining the full email data we can use \textit{get} method). Each element of this list only contains the \textit{id} and \textit{threadId} field.
	
	The parameters of this method are:
	\begin{itemize}
		\item\textit{userId}: user's email address (we can use the special string value \textit{'me'}).
		\item\textit{includeSpamTrash} (optional parameter): boolean parameter which determines if it includes messages with the labels \textit{SPAM} and \textit{TRASH} in the result of the operation.
		\item\textit{labelIds} (optional parameter): it is a list which let us filter the messages by only returning emails with labels that match all of the identifiers that belong to this list.
		\item\textit{maxResults} (optional parameter): an integer which determines the maximum number of messages to return.
		\item\textit{pageToken} (optional parameter): string which specifies a page of results.
		\item\textit{q} (optional parameter): string which let us do an specific query (with the same query format as the Gmail search box) and filter the messages by only returning emails that match with it.
	\end{itemize}
	For knowing the required scopes for invoking this function refer to \cite[/v1/reference/users/messages/list]{gmailAPI}.
	\item\textit{send(userId, body = None, media\_body = None, media\_mime\_type = None)}: it sends the given message to the email addresses specified in the \textit{To}, \textit{Cc} and \textit{Bcc} headers. The first two parameters are the only ones we will use. The first (\textit{userId}) is the user's email address (we can use the special string value \textit{'me'}) and the second is the message we want to send in an RFC 2822 \citep{rfc2822} formatted. For knowing the required scopes for invoking this function refer to \cite[/v1/reference/users/messages/send]{gmailAPI}.
\end{itemize}

\subsection{Threads resource}\label{ssect:threads}
When we access to our inbox, we are actually seeing the inbox threads instead of the messages resource. Every message, even if it is an only email without a reply, is enclosed in a thread resource \citep[/v1/reference/users/threads]{gmailAPI} which is essentially a list, perhaps unitary, of messages resources. In fact, as we can observe in the following resource representation, in its dictionary structure it has a list of messages resources:

\begin{python}
	{
		'id' : string, # The identifier of the thread
		'snippet' : string, # A short part of the text
		'historyId' : unsigned long,
		'messages' : [ users.messages resource ]
	}
\end{python}

The most important methods of this resource are:
\begin{itemize}
	\item\textit{get(userId, id, format = 'full', metadataHeaders = None)}: if successful, this method returns the requested threads resource. In respect of the parameters, they are defined in the same way as in \textit{get} messages resource method (see Section \ref{sssect:msgmethods}) with the exception of the parameter \textit{format}, whose only difference is that it does not accept the \textit{'raw'} value. For knowing the required scopes for invoking this function look up in \cite[/v1/reference/users/threads/get]{gmailAPI}.
	\item\textit{list(userId, includeSpamTrash = False, labelIds = None, maxResults = None, pageToken = None, q = None)}: if successful, it returns a dictionary structure analogous to the view in the \textit{list} message resource method (see Section \ref{sssect:msgmethods}). Needless to say, instead of returning a messages resource list it will give us a threads resource list, which does not contain the complete information of each thread (for example each element of the list has not a list of messages resource). Full thread data can be fetched using the previous method. The parameters of this method are defined in the same way as the \textit{list} messages resource method. For knowing the required scopes for invoking this function refer to \cite[/v1/reference/users/threads/list]{gmailAPI}.
\end{itemize}

\subsection{Drafts resource}\label{ssect:drafts}
The last Gmail API resource we will study is the most easy to understand after knowing all the structures related with emails that we have explained in the above sections: the drafts resource \citep[/v1/reference/users/drafts]{gmailAPI}. Its representation is very simple:

\begin{python}
	{
		'id' : string # The immutable identifier of the draft
		'message' : users.messages resource
	}
\end{python}

As we can observe, a draft is virtually a messages resource with an identifier. Indeed, in order to create a new draft with the \textit{DRAFT} label we must create a MIME message (see Section \ref{sssect:mime}) as we have to do when we want to send a new email by using the \textit{send} messages resource method. Then it is necessary to invoke the drafts resource method \textit{create(userId, body = None, media\_body = None, media\_mime\_type = None)} by giving the value \textit{'me'} and the message created to the first to parameters.

\subsection{API Usage Limits} \label{ssect:apilimits}
One factor to be taken into account is the limitations of the Gmail API \citep[/v1/reference/quota]{gmailAPI} which could become a drawback in the application development. It has a limit on the daily usage and on the per-user rate. In order to measure the usage rate, ``quota units'' are defined depending on the method invoked. In Table \ref{tab:quotaUnits} we can consult the value of some methods in quota units (we have selected the more important methods for our purpose, for the quota units of other methods it is recommended to refer to \cite[/v1/reference/quota]{gmailAPI}).

\begin{table}[h]
	\centering
	\begin{tabular}{|l c r|}
		\hline
		\textbf{Method} & \textbf{Section where the method is explained} & \textbf{Quota units} \\
		\hline\hline
		\textit{getProfile} & \ref{ssect:userres} & 1\\ \hline
		\textit{labels.get} & \ref{ssect:labres} & 1\\ \hline
		\textit{messages.get} & \ref{sssect:msgmethods} & 5\\ \hline
		\textit{messages.list} & \ref{sssect:msgmethods} & 5\\ \hline
		\textit{messages.send} & \ref{sssect:msgmethods} & 100\\ \hline
		\textit{threads.get} & \ref{ssect:threads} & 10\\ \hline
		\textit{threads.list} & \ref{ssect:threads} & 10\\ \hline
		\textit{drafts.create} & \ref{ssect:drafts} & 10\\ \hline
	\end{tabular}
	\caption{Main methods' quota units}
	\label{tab:quotaUnits}
\end{table}

However, both daily usage limit and per-user rate limit are acceptable for the type of software we want to build: 1,000,000,000 quota units per day and 250 quota units per user per second. Therefore there are no constraints (for our purpose) that avoid us to use this API.


\section{Electronic Mail Protocols}
Apart from Gmail API (see Section \ref{sect:gmailapi}), there is another way to read, send emails and access to the user's email data and this is by directly using the communication protocol for electronic mail transmission and the internet standard protocol to retrieve email messages from a mail server over a TCP/IP connection. In this section we are going to study the main email management protocols, both electronic mail transmission protocol (such as Simple Mail Transfer Protocol, which is explained in Section \ref{ssect:smtp}) and message access protocol (such as Internet Message Access Protocol and Post Office Protocol, which are studied in Sections \ref{ssect:imap} and \ref{ssect:pop}, respectively).

In spite of being a mail server-independent solution, as we will see, we are going to find security issues which are going to hinder our user's email data access. These trials come from the automatic server access. The assessment of the advantages and disadvantages of making use of the email protocols or the Gmail API is discussed in Section \ref{ssect:protvsapi}.

\subsection{Simple Mail Transfer Protocol} \label{ssect:smtp}

Simple Mail Transfer Protocol (also known as SMTP) is a network connection-oriented communication protocol used for the exchange of e-mail messages. It was originally defined in \cite{rfc821} (for the transfer) and in \cite{rfc822} (for the message). It is currently defined in \cite{rfc5321} and \cite{rfc5322}. However, this protocol has some limitations when it comes to receiving messages on the destination server. For this reason, this task is intended for other protocols such as the Internet Message Access Protocol (see Section \ref{ssect:imap}) or the Post Office Protocol (refer to Section \ref{ssect:pop}), and SMTP is used specifically to send messages.

Making use of SMTP, an email is ``pushed'' from one mail server to another (next-hop mail server) until it reaches its destination. The message is not routed according to the message recipients specified during the client's connection to the SMTP server, but from the destination mail server. Thanks to the fact that this protocol has a feature to initiate mail queue processing, intermittently connected mail server can extract messages from another remote server when necessary.

\subsection{Post Office Protocol} \label{ssect:pop}

Post Office Protocol (also known as POP) is an application protocol (in OSI Model) for obtaining e-mails stored in a remote Internet server called POP server. It was originally defined in \cite{rfc918} (it was POP version 1, also known as POP1). Current POP version (POP3, in general when we talk about POP we refer to this version) is detailed in \cite{rfc1939}.

POP was designed for receiving emails. Using POP, users with intermittent or very slow Internet connections (such as modem connections) can download their email while online and check it later even when offline. The general operation is: a client using POP3 connects, gets all messages, stores them on the user's computer as new messages, deletes them from the server, and finally disconnects. However some mail clients include the option to leave messages on the server. They use the order UIDL (Unique IDentification Listing) which, unlike most POP3 commands, does not identify messages depending on their mail server ordinal number, due to the fact that this creates problems when a client tries to leave messages on the server, since messages with numbers change from one connection to the server to another. Accordingly, server which makes use of UIDL, assigns a unique and permanent character string to each message. Thus, when a POP3-compatible mail client connects to the server, it uses the UIDL command to map the message identifier. This way the client can use that mapping to determine which messages to download and which to save at the time of download.

Like other old Internet protocols, POP3 used a signature mechanism without encryption. The transmission of POP3 passwords in plain text still occurs. Nowadays POP3 has various authentication methods that offer a diverse range of levels of protection against illegal access to users' mailboxes.

The advantage over other protocols is that between server-client you do not have to send so many commands for communication between them. The POP protocol also works properly if you do not use a constant connection to the Internet or to the network that contains the mail server.

\subsection{Internet Message Access Protocol} \label{ssect:imap}

Internet Message Access Protocol (also known as IMAP) is an application protocol, designed as an alternative to Post Office Protocol (se Section \ref{ssect:pop}) in 1986, which allows the access to stored messages in an Internet server. As with the Post Office Protocol, with IMAP you can access your e-mail from any computer with an Internet connection. The current version of IMAP (IMAP version 4 review 1 or IMAP4rev1) is defined in \cite{rfc3501}.

In contrast to Post Office Protocol, IMAP allows multiple clients to manage the same mailbox. This fact results from the main differences between these two protocols: IMAP does not remove email from server until the client specifically requests it (as POP removes them by default, it is impossible to accessing them from another device which has not the downloaded messages) and it does not download the messages to the user's computer (clients may optionally store a local copy of them). This last property gives raise to several advantages with regard to Post Office Protocol: the immediate notification of the arrival of a mail (due to it works in permanent connection mode) while POP checks if there are new e-mails every few minutes (which causes an appreciable rise in traffic and in the time the user has to wait to send a request to the server, because it is necessary to complete the download of all new messages first), it is possible to create shared folders with other users (it depends on the mail server), the e-mails do not take up memory in the user's local device while POP downloads them regardless of whether they are going to be read or not (effectively IMAP has to download a message when it is going to be read, but they are temporary files and only the e-mail headers are downloaded to manage the mailbox) and it allows the user to manage folders, templates and drafts in server in addition to be able to search a mail from keywords.

\subsection{Advantages and disadvantages of email protocols versus the use of Gmail API} \label{ssect:protvsapi}
