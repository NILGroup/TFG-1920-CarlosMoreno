\chapter{Introduction}
\label{cap:introduccion}

\chapterquote{Have you ever retired a human by mistake?}{Rachael - Blade Runner (1982)}

Smartphone development meant not only a technological advance but a social revolution too. This intelligent telephones have brought with them countless paradigm shifts in terms of the social sphere. Since then, we are able to speak of a new model of human relationship both between people and with our technology. This current relation standard is due to the easy and quick way of accessing the different information that our mobile devices provide us. Long waits (nowadays the meaning of ``long'' waits has changed too, people consider more than two or three second too much time) for obtaining anything such as accessing to a website or showing any operation result, are excessively tedious and could be even frustrating for some smartphone users. When we are using our mobile, we want, as fast as possible, the information we are looking for. Precisely because of this, Human-Computer Interaction (HCI) becomes a very important part in the process of development of most applications, not only in terms of speed of response and efficiency of algorithms, but also in how we show different information and the easiness for obtaining it.

As for the relationships between people, as we have said, they have dramatically changed. There is no doubt that the main driving technologies behind this transformation of our relational paradigm are the social networks and the instant messaging. Focusing on the latter, it is necessary to make a breakdown of what consequences to our interpersonal interaction the instant communication have brought with itself. Just as it happens with the HCI, easiness and speed are probably the first features we look for when we are going to send or receive any information to anybody. If we also expect a reply, the ideal would be to obtain it as quickly as possible. Therefore, in most of occasions, in practice we are looking for an ``automatic'' response from a human, what practically implies that everyone is ``obligated'' to be connected at any time with the answer we are asking for prepared. This new insight into the relationships between people, that perceives the humans as servers who send a request waiting for a quickly reply with the expected data, has promoted a very fast sending of short messages which intends to substitute and simulate an spoken conversation. These little texts are often concise and summarised, and they form an atomic semantic unit, namely they have their own independent meaning.


\section{Incentive}
Introducción al tema del TFM.


\section{Objectives}
Descripción de los objetivos del trabajo.


\section{Working plan}
Aquí se describe el plan de trabajo a seguir para la consecución de los objetivos descritos en el apartado anterior.



\section{Explicaciones adicionales sobre el uso de esta plantilla}
Si quieres cambiar el \textbf{estilo del título} de los capítulos, edita \verb|TeXiS\TeXiS_pream.tex| y comenta la línea \verb|\usepackage[Lenny]{fncychap}| para dejar el estilo básico de \LaTeX.

Si no te gusta que no haya \textbf{espacios entre párrafos} y quieres dejar un pequeño espacio en blanco, no metas saltos de línea (\verb|\\|) al final de los párrafos. En su lugar, busca el comando  \verb|\setlength{\parskip}{0.2ex}| en \verb|TeXiS\TeXiS_pream.tex| y aumenta el valor de $0.2ex$ a, por ejemplo, $1ex$.

TFMTeXiS se ha elaborado a partir de la plantilla de TeXiS\footnote{\url{http://gaia.fdi.ucm.es/research/texis/}}, creada por Marco Antonio y Pedro Pablo Gómez Martín para escribir su tesis doctoral. Para explicaciones más extensas y detalladas sobre cómo usar esta plantilla, recomendamos la lectura del documento \texttt{TeXiS-Manual-1.0.pdf} que acompaña a esta plantilla.

El siguiente texto se genera con el comando \verb|\lipsum[2-20]| que viene a continuación en el fichero .tex. El único propósito es mostrar el aspecto de las páginas usando esta plantilla. Quita este comando y, si quieres, comenta o elimina el paquete \textit{lipsum} al final de \verb|TeXiS\TeXiS_pream.tex|

\subsection{Texto de prueba}

