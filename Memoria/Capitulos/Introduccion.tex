\chapter{Introduction}
\label{cap:introduccion}

\chapterquote{Have you ever retired a human by mistake?}{Rachael - Blade Runner (1982)}

In this chapter we explain the incentives which motivate us to develop this work (see Section \ref{sect:motiv}) and the objectives which we try to fulfil (see Section \ref{sect:obj}) in this work. Finally, the structure of this report is explained in Section \ref{sect:structure}.

\section{Motivation}\label{sect:motiv}
Throughout history, many writers have dedicated their works to other people, such as the author of the Spanish Golden Age Francisco de Quevedo, who dedicated the sonnet ``A una nariz'' to Luis de Góngora and the poem ``A una dama bizca y hermosa'' to an unknown woman; or the poet of the Spanish Generation of 27 Federico García Lorca, who dedicated the elegy ``Llanto por Ignacio Sánchez Mejías'' to his friend and ``Vals en las ramas'' to Vicente Aleixandre. As expected, the style of their works varies according to the addressee of the text (Quevedo does not refer to Góngora in the same way as to his lady). These dedicated texts have travelled from their writers to their recipients in a variety of forms such as letters and newspaper columns. Regardless of the fact that the channel was the same, the sender modified his or her way of speaking to the addressee depending on the relationship established between them. Something similar happens with the e-mail, nowadays established as part of our routine, which turns its users into constant authors who dedicate their sentences to the recipients.

Electronic mail has greatly grown since its appearance in 1971. Indeed it goes on growing and reaches the number of 306 billions of e-mails sent per day. The e-mail has gone beyond the professional scope and has encroached the personal sphere, which means that we send e-mail messages to our contacts which with we keep both a professional and a personal relation. However, there are significant differences in the written text which depend on the receiver of the information even though the channel is the same. Despite having a concise and summarised nature, we do not spend the same amount of time to write every e-mail. Perhaps we are more rigorous with the composition of some messages, we change our writing style or we choose a formal language. This modification of our way of writing a message depending on the recipient is what we study in this work.

The research of the definition of the writing style based on the recipient of the message through mathematical metrics allows us to know how it is necessary to change a text in order to make it appropriate for the audience. Furthermore, the progress in this field of study improves our knowledge about the natural language generation of e-mails, specifically based on the recipient. The automatic personalised writing is able to meet the clear needs of quick responses of a society which is in constant connection with all the information generated every second. For this reason, the progress in this field of study can be useful for the more than four billion e-mail users.

With this motivation, we focus our work on the study of the metrics which describe the writing style of e-mail based on the recipient and how it is possible to take advantage of these mathematical objects in order to build a recipient-based personalised writing system.

\section{Objectives}\label{sect:obj}
The main objective of our work is the proposal of a model for stylometric analysis of e-mails for recipient-based personalised writing. This model must measure the writing style of the user and use this information in order to carry out the composition of an e-mail.

To achieve it, we study the field of stylometry and implement a software which is able to measure the different e-mails from the user's account. Once a system which can extract the messages and calculate a set of style metrics of each one is developed, we have to evaluate the obtained results and study which style markers distinguish better between the different recipients. To obtain this conclusions allow us to suggest a system that can take them into account and generate e-mails with different styles depending on the receiver of the information.

Furthermore, we look for developing reusable tools that can be used in other similar projects. With this in mind, we develop some web services which carry out the different task of extracting and measuring the e-mails. In this way, these tools can be easily incorporated to other implementations.

\section{Report structure}\label{sect:structure}
This work is structured in seven chapters. The first one, as we have seen, is the introduction of the work, where we give an idea of what we explain in the rest of the report. Chapter \ref{cap:estadoDeLaCuestion} (State of the Art) presents the knowledge that we use in the following chapters and which is required for the correct understanding of the work. In it, we explain the communication service which is the centre of our work: the electronic mail, and every aspect of it  which is useful for our work, such as the protocols that defines it. In addition to the electronic mail, we study the field of computational stylometry in Chapter \ref{cap:estadoDeLaCuestion}. Specifically, we delve into the particular characteristics of e-mails as a text document and the most common metrics used for analysed them. This chapter finish with the explanation of the Latent Semantic Indexing, which is an algorithm whose understanding is required for our final model.

The next Chapter (Used Technologies) presents the technologies used for our software development: Gmail API, spaCy, Flask and MongoDB. The first of them is necessary due to the fact that the e-mail account that we analyse belongs to the Gmail service. As for spaCy, it is a syntactic analyser which is useful when we measure the text of the e-mail. In respect of Flask and MongoDB, they are popular software tools used for the development of web services and the management of a NoSQL database, respectively. We use them in our software implementation.

As regards Chapter \ref{cap:analyser} (Style Analyser), in it we explain in detail the developed implementation of the software which is in charge of all the process from the extraction to the measuring of the e-mails of the user. Each of the phases that are part of this task (extraction, preprocessing, typographic correction and measuring), are explained as well as the style metrics used.

After obtaining the metrics' value of each message, we study the set of data in Chapter \ref{cap:analysis} (Style feature analysis), making use of popular machine learning techniques such as K-Means, Principal Component Analysis and Decision Trees. Besides, the previous preparation of the data is explained.

The penultimate chapter (Proposal for a personalised writing model based on the recipient), takes advantage of all the previous knowledge in order to suggest a model for stylometric analysis of e-mails for recipient-based personalised writing and study its viability.

Finally, in Chapter \ref{cap:conclusiones} (Conclusions and Future Work), we expose the conclusions of the developed work and the possible improvements and extensions which can be carried out with the purpose of going on working in the field of study.