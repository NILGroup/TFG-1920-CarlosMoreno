\chapter{Conclusions and Future Work}
\label{cap:conclusiones}

\chapterquote{Difficult to see. Always in motion is the future.}{Yoda - Star Wars: Episode III – Revenge of the Sith (2005)}

After the development of this work, in this chapter we present the conclusions that we can extract from our study. These are explained in Section \ref{sect:conc}. Then, possible options for the continuation of this work are presented in Section \ref{sect:future} in order to follow with the research of the metrics that define the style based on the recipient of the message and take benefit of this field of study to build natural language generation systems that create personalised e-mails.

\section{Conclusions}\label{sect:conc}
Nowadays, electronic mail is a popular communication system both in the professional scene and the personal one. Through it we establish conversations about work, studies and close relationships, among others. However, we do not express an idea to different people in the same way. Depending on our relationship we can vary our vocabulary, expressions, syntactic constructions or formality in our messages with the purpose of transmitting the same idea. In this work we were interested in this modification of the writing style of the same author when the recipient of the e-mail changes. In other words, we were curious to know the stylometric parameters which vary according to the addressee. If we could figure  style metrics out, we would be able to personalise the automatic composition of messages of a natural language generation system.

In order to find out the metrics that define the writing style according to the recipient of the message, it was necessary to obtain a sufficient amount of e-mails. Nevertheless, we have to not only extract them but measure them with a big set of style descriptors. For this reason, we developed a Style Analyser which carries out all the related tasks with the extraction and measurement of the messages of a given user. In particular, we have implemented the process of extracting the different e-mails of the user, preprocessing of the body of each message, correcting the possible typographic mistakes that can appear in the text and measuring the message written by the user.

For the extraction of the e-mails we had to learn about both the protocols and format of electronic mails. Moreover, as the messages that were going to be extracted specifically belonged to a Gmail account, we developed a module able to make use of the Gmail API for the accessing to the user's account information.

Preprocessing a message consists of modifying the e-mail body text with the purpose of having the original message, without the headers and characters introduced by the e-mail service in order to follow the transmission protocol. With this in mind, a preprocessing module was developed as a web service, which allows it to work independently from the rest of the system and be easily reusable in other projects. This type of implementation is repeated in the typographic correction and style measuring modules, which needed to use a syntactic analyser for the success of their tasks. Moreover, for the development of style measuring modules, it was necessary to learn about the different metrics used in the field of computational stylometry and implement them.

Once we had a functional style analyser, we measures the sent message of a user in order to obtain conclusions about the relationship between the implemented metrics and the style used for each recipient. In our data analysis, we concluded that, even though the chosen set of metrics do not differentiate the type of recipients well enough, there are features that describes the writing style better than others and we could select eight of them. We also observed that the obtained results were very dependent on the data, which means that we were limited by the not-so-large number of analysed messages and the unbalanced distribution of types of recipients that the e-mails had.

Many style metrics with more relevance (four out of eight) were related to the variety of different words used in the message. In other words, we are able to claim that the distribution of the vocabulary used plays an important role in the description of writing style based on the recipient. The rest of them took into account features like the relationship between lexical categories (such as verbs and adjectives or determinants and pronouns) and the length of the sentences or words.

Finally, with this information we designed a model of message generation based on the recipient. The system's input is a set of keywords and the e-mail address which is going to receive the message. It makes use of the chosen metrics in order to modify the text until it presents the appropriate values for the recipient under consideration.

It is important to underline that we have not found any research about the writing style based on the recipient or audience of the message whether it is an e-mail or any written text. Likewise, we are able to claim that this work is a first step in this research area and it lays the foundations for natural language generation with style based on the person who is going to receive the information. In particular, it establishes the bases of the recipient-based personalised writing of e-mails.

\section{Future Work}\label{sect:future}
During the analysis of the data obtained after measuring the extracted e-mails, we found some obstacles against the attainment of significant results. One of the most relevant issues that we found is the amount of extracted messages. Since we have implemented most of the modules of the style analyser as web services, it could be easily adapted as a web service with the purpose of being accessible for a bigger amount of people and consequently being able to extract and measure a bigger number of e-mails. Perhaps, this adjustment could require to remove from the analysis process the typographic correction step, otherwise the users would have to correct their messages one by one.

Following with the possible improvements of the style analyser, we could consider (and implement) more style metrics in order to measure the different messages. As we have explained in Section \ref{ssect:stymet}, we can choose between at least a thousand stylistic features. From most simple descriptors, such as Burrow's Delta \citep{burrows2002delta}, to the complex style markers, such as n-grams \citep{brocardo2013authorship}, and e-mail-specific features, such as the set of HTML tags \citep{de2001mining}, we can enlarge our set of metrics. This would allow us to test the variance of each new style descriptor between the different type of recipients.

Once we had measured a big amount of e-mails, we would be ready to carry out a data analysis with more relevant results. It would be appropriate to obtain a balance distribution of the different type of relationship with the recipients, thereby we would have different clusters with a similar number of samples which would allow us to use different machine learning techniques.

Taking advantage of conclusions obtained from this work, its natural extension is the implementation of the proposed model for generating personalised messages based on its recipient (see Chapter \ref{cap:proposal}). As we can remember, it had two phases: searching phase and rewriting phase. The first of it is implemented in the Github repository\footnote{\url{https://github.com/NILGroup/TFG-1920-CarlosMoreno}} (together with the rest of the implemented tools of this work), so it will only be necessary to develop the second module. In its implementation we will have to take into account the most representative metrics that describe the style depending on the recipient of the e-mail. This module could be used for any natural language generation system with the purpose of modifying the style of the generated text.