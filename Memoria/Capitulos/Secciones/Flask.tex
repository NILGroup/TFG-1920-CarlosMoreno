Some modules implemented in this work have been developed as a web service. For this reason, it is necessary to use a framework that helps in the task of programming to easily and efficiently create this type of service. As we have also decided to work in the Python programming language, it is convenient that the framework we choose is developed in that language or is compatible with it. With these restrictions in mind, we chose the Flask\footnote{\url{https://flask.palletsprojects.com/en/1.1.x/}} tool, which allows us to develop free open source web applications written in Python.

Flask is a minimalist framework written in Python that allows us to create web applications quickly and with a minimum number of lines of code. It is designed to make getting started quick and easy, with the ability to scale up to complex applications. Flask is simple and easy to apply in our development, it allows a cleaner backend when handling users, decreasing memory and speed to avoid server failures. It stands out for installing extensions or complements according to the type of project to be developed, that is to say, it is perfect for the rapid prototyping of projects. It includes a web server so we avoid installing one like Apache or Nginx. Its speed is better compared to other similar tools like Django. Generally, Flask's performance is superior due to its minimalist design in its structure. For these reasons we have chosen Flask in order to develop the required web services.