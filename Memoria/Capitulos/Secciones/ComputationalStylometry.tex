\section{Computational stylometry}
This field of Artificial Intelligence (related with the Natural Language Processing and Natural Language Generation) is in charge of studying the writing style in natural language written documents (although it is often use in applications like the detection of plagiarism in programmes). In this section we are going to delve into it in order to known the state of art of this field of study. To achieve this, first a brief introduction is presented (see Section \ref{ssect:introstylo}) and, then, the different applications and techniques used in Computational stylometry are explained (see Section \ref{ssect:techstylo}).

In addition, it will be necessary to explain the presentation of computational stylometry in the specific field of e-mails (see Section \ref{ssect:styloemail}) since, as we can deduce, these present singularities with respect to other types of documents.

Finally, various style writing metrics are going to be explained (see Section \ref{ssect:stymet}) for the purpose of calculating and studying them in the extracted dataset (the entire set of emails that have been extracted).

\subsection{Introduction}\label{ssect:introstylo}
Stylometry \citep{wiki:stylometry} is the application of the study of linguistic style to written language, although it has also been successfully applied to music and painting. It could be defined as the linguistic discipline that applies statistical analysis to literature in order to evaluate the author's style through various quantitative criteria.

According to \cite{stylohist}, the stylometry was born in 1851 when Augustus de Morgan, an English logician, hypothesized that the problem of authorship could be addressed by determining whether one text ``does not deal in longer words'' \citep{morganletters} than another. Following this idea, three decades later, the American physicist Thomas Mendenhall carried out research in which he measured the length of several hundred thousand words from the works of Bacon, Marlowe and Shakespeare \citep{mendenhall1887}. However its results showed that word length is not an effective writing style features which allow us to discriminate between different authors. Since then numerous investigations have been carried out to analyse the parameters that define writing style more precisely.

\cite{neuronalstylometry} defines the writing style as ``a set of measurable patterns which may be unique to an author''. For this reason, various machine learning and statistical techniques have been used to discover the characteristics that determine it. One of the first and most famous successes was the resolution of the controversial authorship of twelve of the Federalist Papers. These documents, a total of eighty-five papers, were published anonymously in 1787 to convince the citizens of New York State to ratify the constitution. They are known to have been written by Alexander Hamilton, John Jay and James Madison, who subsequently claimed their contributions from each of them. However, twelve were claimed both Madison and Hamilton. By using the frequency of occurrence of function words, previously used in \cite{juniusletters}, and employing numerical probabilities adjusted by Bayes' theorem, in \cite{federalistpapers} the twelve papers disputed were attributed to James Madison. Thereafter, Federalist Papers is a famous example in this area for testing the different solutions, as it happens in \cite{neuronalstylometry}, which make use of neural networks to solve this problem.

\subsection{Applications and techniques}\label{ssect:techstylo}
In addition to the detection and verification of authorship in historical, literary and even forensic investigations, stylometry is used in other areas such as the detection of fraud and plagiarism, the classification of documents according to their genre or audience, etc. Other possible applications of this area are the prediction of the gender, age or personality of the author as it happens in \cite{schwartz2013personality} and the natural language generation with Style \citep[Section 5.1]{nlgsoa}.

To address all these problems, mostly statistical techniques are used. Some of them, which are more complex, are more recognized for belonging to the field of machine learning such as neural networks and Principal Components Analysis \citep{PCAstyle}, while others are based on merely syntactic-statistical concepts as in the well-known software implementations such as stylo \citep{stylor} and STYLENE \citep{stylene}. To this last type also belongs techniques based on dictionary word counting using Linguistic Inquiry and Word Count also known as LIWC \citep{liwc2015}, while more recent ones which use simple lexico-syntactic patterns, such as n-grams and part-of-speech (POS) tags \citep{mihalcea2009lie, ott2011finding}, belongs to the machine learning. We can also find techniques outside this paradigm, such as the writing style features driven from Context Free Grammar (CFG) as we can observe in \cite{cfgstylo}.

In order to address our work, we are going to make use of writing style metrics (which will be explained in Section \ref{ssect:stymet}), based on simple statistics like the mean and easy probabilistic metrics like the entropy.

\subsection{Style in emails}\label{ssect:styloemail}


\subsection{Style metrics}\label{ssect:stymet}