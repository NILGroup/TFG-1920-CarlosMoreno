As we will see, we will need to store different types of data during the analysis of the e-mails. For this task we have chosen MongoDB\footnote{\url{https://www.mongodb.com/}} which is an open source, document-oriented, NoSQL database system.

Instead of storing data in tables, as is done in relational databases, MongoDB stores BSON data structures (a specification similar to JSON) with a dynamic schema, making data integration in certain applications easier and faster \citep{gyHorodi2015comparative}. This feature is perfectly adapted to our needs since, as we will see, the data structures we will handle will be variable. In addition, no powerful resources required to work with it and, thanks to the flexibility offered by being a NoSQL database, we can easily carry out modifications in our conceptual model of the database without having to worry about problematic changes between primary and foreign keys between tables. Moreover, it has official drivers for the Python programming language we will be working on.