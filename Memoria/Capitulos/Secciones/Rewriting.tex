The rewriting phase will be responsible for modifying the message, obtained by the search phase, as necessary so that it has the style corresponding to the final recipient of the e-mail. To achieve this, we will need to know the category to which the person who will receive the message belongs. For this purpose, we will consult their e-mail address in the database where we can find the classification of the different contacts. In case no information is found about the consulted address, it will be necessary for the user to provide the category to which he or she belongs. As we will explain, this system requires to have previously data (written emails from which their style metrics have been extracted) of the category to which the recipient belongs, which can be a problem in case we consider writing a message destined to a new category.

As we have explained (see Chapter \ref{cap:analysis}), there are eight style metrics of the initial twenty-eight that best describe the writing style depending on the recipient of the message. These style markers are: \textit{verbAdjectiveRatio} (it is obtained by dividing the number of verbs by the number of adjectives), \textit{detPronRatio} (it is obtained by dividing the number of determinants by the number of pronouns), \textit{meanSentLen} (it is the average sentence length in word count), \textit{meanWordLen} (it is the average word length in number of characters), \textit{richnessYule} (it depends on the diversity of words, i.e. the number of different words and the number of words we do not repeat or appear twice, three, etc.), \textit{difficultyLevel} (it depends on both the percentage of words with one or two characters and the \textit{meanSentLen}), \textit{stopRatio} (it is the percentage of stop words) and \textit{entropy} (it depends on the number of times the same word appears). The modification of the e-mail will be based on trying to vary the value of these features according to the category to which the recipient belongs. In this way, we will obtain a message with values close to the averages of the style metrics of the category under consideration. There are several methods (with which we must assume that the message generated may not be correct due to issues such as polysemy and concordance in gender and number, among others) to modify them:

\begin{itemize}
	\item \underline{Change the number of adjectives}: removing adjectives is a simple task to perform and, except in cases where the adjective differentiates one entity from another, it does not cause problems when modifying the text. However, adding them is slightly more complicated. For this purpose, we could use a corpus of n-grams (like the Google n-grams\footnote{\url{http://storage.googleapis.com/books/ngrams/books/datasetsv2.html}}) to write the adjective that most commonly accompanies the noun at hand. Another way to address this problem is to add the adjectives according to the frequency with which the user uses them (using techniques such as probabilistic grammars used by \cite{halliday2014corpus}) making use of the stored messages. The disadvantage of the latter method is that it requires a large number of e-mails and with one as small as ours, it is likely that good results would not be obtained. On the other hand, this solution guarantees the use of the user's \textit{lexicon} (set of words used). The modification of the number of adjectives, will allow us to vary the following style metrics: \textit{verAdjectiveRatio} (although changing the number of verbs can be a complex task and we can find many problems, as we have seen, changing the number of adjectives is feasible), \textit{menaSentLen}, \textit{difficultyLevel} (as it affects the average length of sentences), \textit{richnessYule} (as it depends on the number of different words and the amount of times each word appears), \textit{stopRatio} (as it adds or removes adjectives, the percentage of stop words will be modified) and \textit{entropy} (changing the number of words in the text also changes the probability of each word appearing).
	
	\item\underline{Substitute words for synonyms}: Although we may make mistakes in cases such as polysemic words, replacing words with their synonyms would allow us to increase or decrease the value of some style metrics. To obtain the corresponding synonyms there are many web services\footnote{such as \url{https://holstein.fdi.ucm.es/nil-ws-api/}} or corpus from which we can extract them. Besides, we can use our bag of words (\textit{wordsAppearance} style marker) in order to use synonyms which belong to the user's \textit{lexicon}. The style metrics that would change their value with this method would be: \textit{meanWordLen} (it is possible to replace some words with longer or shorter synonyms to modify this feature), \textit{difficultyLevel} (as it also depends on the number of words in a syllable, although in our implementation it is the number of words with one or two characters, the replacement by synonyms of greater or lesser length can vary this descriptor), \textit{richnessYule} and \textit{entropy} (if a word is replaced by a synonym, its probability of occurrence decreases and that of the synonym used increases).
	
	\item\underline{Change the number of adverbs}: the elimination of adverbs may not be as easy as in the case of adjectives, as these express circumstances, such as mode, place, time, quantity, affirmation, doubt, etc. Nevertheless, it is possible to add or remove adverbs of quantity or similar (such as very, little or quite). To carry out this task we can use the same methods we used with adjectives: corpus of n-grams or reusing the ones written by the user in the analysed e-mails. This modification of the text would affect the following metrics: \textit{meanSentLen}, \textit{difficultyLevel}, \textit{richnessYule}, \textit{stopRatio} and \textit{entropy}.
	
	\item\underline{Change the number of pronouns}: When trying to remove or add pronouns we will be faced with the problem of co-reference, which consists of knowing to which entity each of the pronouns in the text refer. Nowadays we find some models to solve this challenge with quite promising results. The solutions use all kinds of techniques, such as the neural net scoring model that spaCy has\footnote{\url{https://spacy.io/universe/project/neuralcoref}} (which is an implementation of the study of \cite{clark2016deep}). Replacing an entity with a pronoun (i.e. reducing the number of pronouns) is not a complicated task, it just requires taking into account parameters, such as gender or number, that are offered by syntactic analysers such as spaCy. On the other hand, the opposite task involves the co-reference problem mentioned. One possible solution is to use existing complex systems such as the one we have presented. Another possibility is to take advantage of the characteristics of e-mails to obtain co-reference results to text pronouns. As e-mails are not very complex texts and do not usually involve many entities, it is possible to obtain a large percentage of successes by looking for nouns that have the same number and gender and staying with the most numerous or the closest to the pronoun. In any case, the modification of the number of pronouns in the text would affect the following style metrics: \textit{detPronRatio}, \textit{meanSentLen}, \textit{stopRatio}, \textit{difficultyLevel}, \textit{richnessYule} and \textit{entropy}.
\end{itemize}

With these modifications of the original message, it will be possible to bring its style metrics closer to the desired value. However, it is necessary to underline that each one of these affects more than one style markers, which means that we are significantly varying the value of more than one metric. In the presentation of this model, we are assuming, as the logic indicates, that all these descriptors are slightly correlated, either in directly or inversely proportional way. Nevertheless, we run the risk of making use of one of the four previous changes and approaching the desired value of a feature while we are moving away from the mean of other style metric.

If we are able to change the values of the eight chosen style markers to its corresponding mean according to the category of the contact, we will have a text of the personalized message based on the recipient as we wanted to obtain.