\chapter*{Abstract}

\section*{\tituloPortadaEngVal}

Nowadays, more than 306 billion e-mails are sent daily, both in the professional and personal scopes. However, despite the fact that the channel is the same, our style varies depending on the recipient of the message. Stylometry in e-mails is a recent field of study that tries to obtain the definition of writing style through metrics. Nevertheless, most research in this field focuses on spam detection or message author identification and authentication. We propose a new approach: to study the style depending on the recipient of the e-mail. Moving in this direction would allow us to personalise e-mail writing systems so that they are capable of generating different messages depending on the recipient.

In this work we will develop a tool for the stylometric analysis of e-mails, for the Gmail service, which will allow us to extract and calculate different metrics from the messages of a user. This style analyser will have four modules (extraction, preprocessing, typographic correction and style measuring) that will deal with the different phases needed to obtain the style descriptors of each of the messages.

Once we have the results of evaluating the different metrics on each message, we will analyse the results obtained. To this end, we will use popular machine learning techniques such as K-Means, Principal Component Analysis and Decision Trees. The objective will be to draw conclusions that will allow us to propose a model of stylometric analysis of e-mails for personalized writing based on the recipient. In this data analysis we will find eight metrics that better distinguish style according to the receiver of the information.

Finally, we will present the design of a system that uses these eight metrics to write different e-mails according to the recipient. This model could be useful to customize those natural language generation systems depending on the recipient, or perhaps on the audience to which the text is addressed.

\section*{Keywords}

\noindent Stylometry, e-mail, metric, style, recipient, Gmail, style analyser, machine learning, personalised writing, Latent Semantic Indexing.



