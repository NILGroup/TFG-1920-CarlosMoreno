% +--------------------------------------------------------------------+
% | Acknowledgements Page (Optional)                                   |
% +--------------------------------------------------------------------+

\chapter*{Agradecimientos}

Sin lugar a dudas, este trabajo no habría salido adelante sin la excepcional dirección de Raquel Hervás y Gonzalo Méndez. Aunque no he tenido el placer de disfrutar de su labor como profesores, he tenido la oportunidad de ser dirigido por ellos en este Trabajo de Fin de Grado lleno de anécdotas (incluso culinarias sobre croquetas) en un año de todo menos corriente. No solo es de elogiar su gigantesca dedicación, esfuerzo y sus incansables espíritus de dar lo mejor de ellos mismos tanto a nivel profesional como personal, sino también su gran cercanía, empatía y predisposición para ayudar siempre que lo necesité. Nunca olvidaré la gran cantidad de reuniones (que hacían el esfuerzo de programar en sus apretadísimas agendas y dedicarme todo el tiempo necesario) que tuvimos en sus despachos (o cada uno en sus casas con simpáticos invitados sorpresa) en las que, además de salir con más ilusión que la que tenía al entrar, conseguía evadirme de los problemas y de la jerarquía entre profesor y estudiante gracias al buen ambiente que creaban (si no era lunes). Son dos personas brillantes en lo académico, profesional y personal que fueron capaces de sacar lo mejor de mí y hacerme más llevadero académica y anímicamente este último curso. Quizás no consigamos ``desbancar a Gmail'' como bromeábamos (aunque toda broma tiene su parte de verdad), pero logramos un trabajo del que me siento orgulloso y contento de haberlo compartido con ellos.

Otra persona que me ha ayudado durante el desarrollo de este Trabajo de Fin de Grado es Antonio, a quien le agradezco mucho el haberme echado una mano con la comunicación con el servidor y el haber dedicado tanto tiempo y esfuerzo a montar el contenedor donde he podido alojar el desarrollo realizado.

Mis compañeros y amigos de carrera, de la Delegación de estudiantes y de la universidad en general también comparten conmigo este trabajo, pues juntos, ya sea por empatía o de facto, hemos disfrutado y sufrido (bueno, me han sufrido) en las alegrías y desventuras no solo de este curso y del Trabajo de Fin de Grado, sino de todo mi recorrido académico en esta universidad, que a veces parecía que duraba 27 años y, otras veces, 27 segundos.

También debo agradecerle el poder presentar este trabajo a todas esas personas que diariamente han empatizado con mis quejas y sonrisas durante todos mis estudios, como mi hermano Luis (la pata científica de este estudio), Irene (la pata literaria de este trabajo), mis tíos Luis y Montse (quienes siempre han estado con sus geniales consejos y buenos ánimos) y mi padre. Todas las personas que se han preocupado y me han apoyado incondicionalmente durante toda esta etapa de mi vida se merecen un agradecimiento tan grande que nunca serán suficientes ``gracias'' que dar.

Por último, no puedo acabar sin hacer una especial mención a la persona que se ha agobiado y preocupado mucho más que yo en todos y cada uno de mis exámenes, trabajos y pruebas de evaluación. Una persona a la que no solo le debo la vida, sino el haberme dado la oportunidad de recibir la mejor educación posible y de formarme para poder salir adelante el día de mañana: mi madre. A ella no le dedico y agradezco mi Trabajo de Fin de Grado, sino toda mi carrera.