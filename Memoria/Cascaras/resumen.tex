\chapter*{Resumen}

\section*{\tituloPortadaVal}

A lo largo de la historia son muchos los escritores que destinan sus obras a otras personas, como el autor del Siglo de Oro Francisco de Quevedo, quien dedicó el soneto \textit{A una nariz} a Luis de Góngora y el poema \textit{A una dama bizca y hermosa} a una desconocida mujer; o el poeta de la generación del 27 Federico García Lorca, quien dedicó la elegía \textit{Llanto por Ignacio Sánchez Mejías} a su amigo y \textit{Vals en las ramas} en homenaje a Vicente Aleixandre. Como es de esperar, el estilo de sus obras varía en función del destinatario de la misma (Quevedo no se refiere de la misma forma a Góngora que a su dama). Lo mismo ocurre con el correo electrónico, hoy en día instaurado como parte de nuestra rutina, que nos convierte a sus usuarios en constantes autores que dedican sus oraciones a los destinatarios. Con él nos comunicamos diariamente tanto en el ámbito profesional como en el personal, pero, aunque el canal sea el mismo, la forma de redactar el mensaje varía. Medir este cambio en el estilo de escritura, nos permitiría personalizar en función del destinatario los sistemas de redacción de correos electrónicos.

En nuestro trabajo, estudiamos cómo cambia nuestro estilo de escritura en función del receptor del mensaje a través de distintas métricas. Para lograrlo, desarrollaremos un sistema de extracción y medición de los mensajes del usuario, cuyos resultados analizaremos posteriormente. Con las conclusiones de nuestro análisis, propondremos un modelo de análisis estilométrico de correos electrónicos para la redacción personalizada basada en el destinatario.

\section*{Palabras clave}
   
\noindent Máximo 10 palabras clave separadas por comas

   


