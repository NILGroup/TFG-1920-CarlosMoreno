\chapter*{Resumen}

\section*{\tituloPortadaVal}

Hoy en día se envían más de 306 mil millones de correos electrónicos diarios tanto en el ámbito profesional como el personal. Sin embargo, a pesar de que el canal sea el mismo, nuestro estilo varía en función del destinatario del mensaje. La estilometría en correos electrónicos es un campo de estudio reciente que trata de parametrizar el estilo de escritura a través de métricas. La mayoría de las investigaciones en este campo se centran en la detección de spam o identificación y autenticación de la autoría de los mensajes. En este trabajo se plantea un nuevo enfoque: estudiar el estilo dependiendo del destinatario del correo electrónico. El avance en esta dirección permitiría personalizar los sistemas de redacción de correos electrónicos de manera que fueran capaces de generar mensajes distintos en función del destinatario.

En este trabajo se desarrolla una herramienta de análisis estilométrico de correos electrónicos, para el servicio de Gmail, que permite extraer y calcular distintas métricas de los mensajes de un usuario. Dicho analizador de estilo cuenta con cuatro módulos (extracción, preprocesamiento, corrección tipográfica y medición de estilo) que abordan las distintas fases necesarias para obtener los descriptores de estilo de cada uno de los mensajes.

Una vez se cuenta con los resultados al evaluar las distintas métricas sobre cada mensaje, se analizan. Para ello se hace uso de populares técnicas de aprendizaje automático como K-Medias, Análisis de Componentes Principales y Árboles de Decisión. El objetivo es extraer conclusiones que permitan proponer un modelo de análisis estilométrico de correos electrónicos para la redacción personalizada basada en el destinatario. En este análisis de datos se encuentran ocho métricas que distinguen mejor el estilo en función del receptor de la información.

Por último, se presenta el diseño de un sistema que utiliza estas ocho métricas para redactar correos electrónicos distintos según el destinatario. Este modelo puede ser de utilidad para personalizar aquellos sistemas de generación de lenguaje natural en función del destinatario, o de la audiencia a la que va dirigida el texto.

\section*{Palabras clave}
   
\noindent Estilometría, correo electrónico, métrica, estilo, destinatario, Gmail, analizador de estilo, aprendizaje automático, redacción personalizada, Latent Semantic Indexing.

   


